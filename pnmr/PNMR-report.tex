\documentclass{article}
\usepackage[utf8]{inputenc}
\usepackage{amsmath}
\usepackage[margin=1in]{geometry}
\usepackage{changepage}
\usepackage{multicol}
\usepackage{ragged2e}
\renewcommand{\thesection}{\Roman{section}}
\renewcommand{\thesubsection}{\thesection.\roman{section}}
\setlength{\columnsep}{1cm}
\setlength\parindent{5pt}
\begin{document}
%\begin{multicols}{1}
\Centering
\Large{\textbf{Pulsed NMR Experiment on Mineral Oil, CuSO\textsubscript{4} and 
Isoproponyl-2 samples}} \\
\small{Herbert D. Ludowieg \\
\begin{adjustwidth}{0.5in}{0.5in}
\justify
Nuclear Magnetic resonance has been a tool that has been used by scientists for
many years to analyze solids and liquids to identify structure and intrinsic 
properties of such materials. In this experiment, the Spin-Lattice relaxation 
time and Spin-Spin Relaxation time for mineral oil, copper sulfate solutions 
and Isoproponyl-2 were measured. The Spin-Lattice relaxation times of mineral 
oil and Isoproponyl-2 solution were measured to be: 37.7 $\pm$ 0.9 ms, 570 
$\pm$ 30 ms respectively. The Spin-Lattice relaxation times of the CuSO
\textsubscript{4} are as follows: 0.005M solution 99 $\pm$ 8 ms, 0.010M 
solution 61 $\pm$ 2 ms, 0.050M solution 13.6 $\pm$ 0.2 ms, 0.100M solution
6.5 $\pm$ 0.1 ms, 0.200M solution 3.26 $\pm$ 0.06 ms, 0.500M solution 1.25 
$\pm$ 0.03 ms, 1.000M solution 0.69 $\pm$ 0.01 ms. For the Spin-Spin relaxation 
times of mineral oil the results are as follows: 58.2 $\pm$ 0.4 ms with 
individual measurements, 27.1 $\pm$ 0.9 ms with Carr-Purcell method and 54.8 
$\pm$ 0.4 with the Meiboom-Gill method. For Isoproponyl-2 the Spin-Spin 
relaxation time was measured to be 174 $\pm$ 3 ms with the Meiboom-Gill method. 
For the CuSO\textsubscript{4} solutions the Meiboom-Gill method was employed 
and the Spin-Spin relaxation times are as follows: 0.005M solution 98 $\pm$ 3 
ms, 0.010M solution 79 $\pm$ 1 ms, 0.050M solution 17.3 $\pm$ 0.6 ms, 0.100M 
solution 9.6 $\pm$ 0.3 ms, 0.200M solution 4.5 $\pm$ 0.1 ms, 0.500M solution 
1.74 $\pm$ 0.05 ms and 1.000M solution 0.93 $\pm$ 0.02 ms.
\end{adjustwidth}%\end{multicols}
\begin{multicols}{2}
\justify
\section{Introduction}
Nuclear Magnetic Resonance (NMR) is a spectroscopic method that was developed 
by Edward Purcell and Felix Bloch independently with different instrumentation 
in the late 1940's. This new spectroscopic technique made use of the nuclear 
magnetic moments of atoms to be able to tell structure among other intrinsic 
properties of solids and liquids. This discovery marked a landmark in the 
study of materials by many different disciplines in scince including: physics, 
chemistry and biology. 
\\
Originally the method was developed with taking measurements of the Free 
Induction Decay (FID) signal. This was a signal that could be measured when 
a population of spins aligned in a magnetic filed given by a permanent 
magnetic field (B\textsubscript{0}) were disturbed from that position along 
the z-axis by a Continuous Wave (CW) Radio Frequency (RF) magnetic field 
(B\textsubscript{1}) along the xy-plane of the sample spins. The decay of the 
spins back to their equilibrium position along the z-axis became known as the 
FID signal. By analyzing this signal different properties of the material 
being tested would be found such as the Spin-Spin and the Spin-Lattice 
relaxation times.
\\
However, in 1950 Edward L. Hahn had discovered that after the sample spins were excited by the B\textsubscript{1} pulse there would be a secondary pulse on the
FID signal at resonance. It was noticed that this secondary would not be in 
response to the previous pulse which was a strong RF pulse lasting for a given 
amount of time. This would be known as the Hahn Spin Echo. This method has 
become widely used and is the basis for the measurements in this experiment.
\\
NMR has also made its way into medicine in Magnetic Resonance Imaging (MRI) 
where it has offered a better alternative of imaging tissues by utilizing ions 
and other substances that can be imaged with NMR. It has become a better 
alternative since there is no longer the requirement of a radioactive sample 
being injected into the patient like in CAT scans where the patient is exposed 
to X-Rays. The only downside is that there can be no magnetic materials present 
or it can damage the high magnetic field magnet or the patient.
\section{Theory}
NMR makes use of the uneven spin configurations of protons and neutrons which, 
like with electrons they are a spin 1/2 particle and arrange themselves in 
a configuration called the Nuclear Shell model which is depicted in a very 
similar fashion to that of the Atomic Shell model which describes the 
arrangement 
of electrons in the electron cloud. This is to say that they too fill up their 
respective shell positions respecting the Pauli Exclusion principle. So the 
first problem to resolve was how to make sure that there would be enough of a 
population density of spins pointing in a similar direction. This was resolved 
by placing the sample of material to be measured in a B\textsubscript{0} field 
strong enough to be able to align the spins
spins along a similar axis which will be the z-axis. By placing a 
second magnet on the perpendicular plane (xy-plane) in a Helmholtz 
configuration a the second CW-RF field could be created. When current at the 
resonant frequency of the sample being tested was passed thorugh this coil it 
would 
\\
However, in 1950 Edward L. Hahn had discovered that after the sample spins were 
excited by the B\textsubscript{1} pulse there would be a secondary pulse on the 
FID signal at resonance. This would occur when a strong RF pulse was applied on 
the sample 
\end{multicols}
}
\end{document}
