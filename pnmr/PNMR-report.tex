\documentclass{article}
\usepackage[utf8]{inputenc}
\usepackage{amsmath}
\usepackage[margin=1in]{geometry}
\usepackage{changepage}
\usepackage{multicol}
\usepackage{ragged2e}
\usepackage{hyperref}
\renewcommand{\thesection}{\Roman{section}}
\renewcommand{\thesubsection}{\thesection.\roman{section}}
\setlength{\columnsep}{1cm}
\setlength\parindent{5pt}
\begin{document}
%\begin{multicols}{1}
\Centering
\Large{\textbf{Pulsed NMR Experiment on Mineral Oil, CuSO\textsubscript{4} and 
Isoproponyl-2 samples}} \\
\small{Herbert D. Ludowieg \\
\begin{adjustwidth}{0.5in}{0.5in}
\justify
Nuclear Magnetic resonance has been a tool that has been used by scientists for
many years to analyze solids and liquids to identify structure and intrinsic 
properties of such materials. In this experiment, the Spin-Lattice relaxation 
time and Spin-Spin Relaxation time for mineral oil, copper sulfate solutions 
and Isoproponyl-2 were measured. The Spin-Lattice relaxation times of mineral 
oil and Isoproponyl-2 solution were measured to be: 37.7 $\pm$ 0.9 ms, 570 
$\pm$ 30 ms respectively. The Spin-Lattice relaxation times of the CuSO
\textsubscript{4} are as follows: 0.005M solution 99 $\pm$ 8 ms, 0.010M 
solution 61 $\pm$ 2 ms, 0.050M solution 13.6 $\pm$ 0.2 ms, 0.100M solution
6.5 $\pm$ 0.1 ms, 0.200M solution 3.26 $\pm$ 0.06 ms, 0.500M solution 1.25 
$\pm$ 0.03 ms, 1.000M solution 0.69 $\pm$ 0.01 ms. For the Spin-Spin relaxation 
times of mineral oil the results are as follows: 58.2 $\pm$ 0.4 ms with 
individual measurements, 27.1 $\pm$ 0.9 ms with Carr-Purcell method and 54.8 
$\pm$ 0.4 with the Meiboom-Gill method. For Isoproponyl-2 the Spin-Spin 
relaxation time was measured to be 174 $\pm$ 3 ms with the Meiboom-Gill method. 
For the CuSO\textsubscript{4} solutions the Meiboom-Gill method was employed 
and the Spin-Spin relaxation times are as follows: 0.005M solution 98 $\pm$ 3 
ms, 0.010M solution 79 $\pm$ 1 ms, 0.050M solution 17.3 $\pm$ 0.6 ms, 0.100M 
solution 9.6 $\pm$ 0.3 ms, 0.200M solution 4.5 $\pm$ 0.1 ms, 0.500M solution 
1.74 $\pm$ 0.05 ms and 1.000M solution 0.93 $\pm$ 0.02 ms.
\end{adjustwidth}%\end{multicols}
\begin{multicols}{2}
\justify
\section{Introduction}
Nuclear Magnetic Resonance (NMR) is a spectroscopic method that was developed 
by Edward Purcell and Felix Bloch independently with different instrumentation 
in the late 1940's. This new spectroscopic technique made use of the nuclear 
magnetic moments of atoms to be able to tell structure among other intrinsic 
properties of solids and liquids. This discovery marked a landmark in the 
study of materials by many different disciplines in scince including: physics, 
chemistry and biology. 
\\
Originally the method was developed with taking measurements of the Free 
Induction Decay (FID) signal. This was a signal that could be measured when 
a population of spins aligned in a magnetic filed given by a permanent 
magnetic field (B\textsubscript{0}) were disturbed from that position along 
the z-axis by a Continuous Wave (CW) Radio Frequency (RF) magnetic field 
(B\textsubscript{1}) along the xy-plane of the sample spins. The decay of the 
spins back to their equilibrium position along the z-axis became known as the 
FID signal. By analyzing this signal different properties of the material 
being tested would be found such as the Spin-Spin and the Spin-Lattice 
relaxation times.
\\
However, in 1950 Edward L. Hahn had discovered that after the sample spins were excited by the B\textsubscript{1} pulse there would be a secondary pulse on the
FID signal at resonance. It was noticed that this secondary would not be in 
response to the previous pulse which was a strong RF pulse lasting for a given 
amount of time. This would be known as the Hahn Spin Echo. This method has 
become widely used and is the basis for the measurements in this experiment.
\\
NMR has also made its way into medicine in Magnetic Resonance Imaging (MRI) 
where it has offered a better alternative of imaging tissues by utilizing ions 
and other substances that can be imaged with NMR. It has become a better 
alternative since there is no longer the requirement of a radioactive sample 
being injected into the patient like in CAT scans where the patient is exposed 
to X-Rays. The only downside is that there can be no magnetic materials present 
or it can damage the high magnetic field magnet or the patient.
\section{Theory}
NMR experiments can actually work because much like electrons, nucleons 
(protons, neutrons) have spin $\pm1/2$ and arrange themselves in 
a configuration called the Nuclear Shell model which is depicted in a very 
similar fashion to that of the Atomic Shell model which describes the 
arrangement 
of electrons in the electronic orbitals. This is to say that they too fill up 
their 
respective shell positions according to the Pauli Exclusion principle. By 
having a very strong permanent magnetic field B\textsubscript{0} the nuclear 
spins could be alinged in the direction of the magnetic field (z-axis). With 
magnetic energy given by,
\begin{equation} %1
U = -\vec{\mu}\cdot\vec{B}
\end{equation}
Where, $\vec{\mu}$ is defined by,
\begin{equation} %2
\vec{\mu} = \gamma\hbar\vec{I}
\end{equation}
Where, $\vec{J} = \hbar\vec{I}$ is the angular momentum and $\vec{\mu}$ is the 
magnetic moment of the particle. With $\gamma$ being the gyromagnetic ratio, 
$\vec{I}$ is the spin of the particle which is $\pm1/2$ in our case. Therefore, 
the energy difference between the two energy levels becomes,
\begin{equation} %3
\Delta U = \hbar\omega_0 = \gamma\hbar B_0
\end{equation}
Where, $B_0$ is the permanent magnetic field aligned with the z-axis and 
$\omega_0 = \gamma B_0$ which gives the resonance value for a proton. In which 
the gyromagnetic ratio is equal to $2.675 \cdot 10^4 rad/s\cdot gauss$
\cite{ref:2}.
\\
According to equation 3, the difference in the energy of the two spins is 
highly controlled by the permanent magnetic field that is used. Therefore, the 
energy difference between the two spins is very small and there is an almost 
equal probability of the spins being in one direction or the opposite when in 
the magnetic field. The Boltzman Distribution gives us the ratio of the spin 
populations as such,
\begin{equation}
\frac{N_1}{N_2} = e^{\frac{-\Delta U}{kT}} = e^{\frac{-h\omega_0}{kT}}
\end{equation}
Where, there will be a net spin population of approximately 1 in 10,000 in any 
of the directions. This is one reason for which NMR requires high number of 
nuclei present in the sample\cite{ref:1}.
However, it's almost as likely that they will line up against the B
\textsubscript{0} since the difference between the spin 1/2 and spin -1/2 are 
only a few milicalories apart. For this fact there is a thermal equilibrium of 
the two spin populations which follows the Boltzman distribution law. Which 
tells us that the population difference between these two states is less than 
1 nucelus in 10,000. According to this, a very large sample will be needed in 
order to be able to make good measurements with NMR\cite{ref:1}.
\\
The fundamental principle behind NMR centers on the induction of the transition 
between the different Zeeman splitting levels of a nucelus. These splitting 
levels are important since they are the excited spin states of the nucleons and 
as they relax back to their ground state configuration they will create the 
FID signal. To be able to transition between the different spin states an 
additional field to B\textsubscript{0} must be applied. This field unlike the 
B\textsubscript{0} field must be a variable radio frequency (RF) field known as 
the B\textsubscript{1} field and it must have a frequency equal to the resonant 
frequency of the nucleons in the sample. It should also be noted that the 
transitions obey the transition rules of the angular momenta, that is to say, 
the difference in the angular momenta between the transitions cannot exceed 1.
\\
To be 
be able to make measurements on the sample a second magnetic field on the 
perpendicular plane (xy-plane) in a Helmholtz configuration creating a 
B\textsubscript{1} field. It is only on this plane where measurements can 
be taken as the magnetic moments will precess about the z-axis when on the 
xy-plane. As the magnetic moments decay back to equilibrium the FID is measured.
\\
NMR can only be employed on atoms that have a net spin. One example of this 
such an atom is \textsuperscript{13}C which has 6 protons and 7 neutrons. 
However, in 1950 Edward L. Hahn had discovered that after the sample spins were 
excited by the B\textsubscript{1} pulse there would be a secondary pulse on the 
FID signal at resonance. This would occur when a strong RF pulse was applied on 
the sample 
\begin{thebibliography}{9}
\bibitem{ref:1}
Paudler, W. W. (1987) \emph{Nuclear Magnetic Resonance General 
Concepts and Applications}. John Wiley \& Sons Inc.
\bibitem{ref:2}
PHY 408 Lab manual. \emph{Pulsed NMR}
\end{thebibliography}
\end{multicols}
}
\end{document}
