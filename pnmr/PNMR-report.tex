\documentclass{article}
\usepackage[utf8]{inputenc}
\usepackage{amsmath}
\usepackage[margin=1in]{geometry}
\usepackage{changepage}
\usepackage{multicol}
\usepackage{ragged2e}
\setlength{\columnsep}{1cm}
\setlength\parindent{5pt}
\begin{document}
%\begin{multicols}{1}
\Centering
\textbf{Pulsed NMR Experiment on Mineral Oil, CuSO\textsubscript{4} and 
Isoproponyl-2 samples} \\
\small{Herbert D. Ludowieg \\
\begin{adjustwidth}{0.5in}{0.5in}
\justify
Nuclear Magnetic resonance has been a tool that has been used by scientists for
many years to analyze solids and liquids to identify structure and intrinsic 
properties of such materials. In this experiment, the Spin-Lattice relaxation 
time and Spin-Spin Relaxation time for mineral oil, copper sulfate solutions 
and Isoproponyl-2 were measured. The Spin-Lattice relaxation times of mineral 
oil and Isoproponyl-2 solution were measured to be: 37.7 $\pm$ 0.9 ms, 570 
$\pm$ 30 ms respectively. The Spin-Lattice relaxation times of the CuSO
\textsubscript{4} are as follows: 0.005M solution 99 $\pm$ 8 ms, 0.010M 
solution 61 $\pm$ 2 ms, 0.050M solution 13.6 $\pm$ 0.2 ms, The methods used for the Spin-Spin relaxation 
time were by reading the individual Hahn-Echo signals, Carr-Purcell method and 
the Meiboom-Gill method. 
\end{adjustwidth}%\end{multicols}
\begin{multicols}{2}
\end{multicols}
}
\end{document}
