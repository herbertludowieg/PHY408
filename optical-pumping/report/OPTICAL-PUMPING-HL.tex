\documentclass[twocolumn]{article}
\usepackage{caption}
\usepackage[utf8]{inputenc}
\usepackage{amsmath}
\usepackage[margin=1in]{geometry}
\usepackage{changepage}
\usepackage{titling}
\usepackage{ragged2e}
\usepackage{abstract}
\usepackage{enumitem}
\usepackage{graphicx}
\usepackage{hyperref}
\renewcommand{\thesection}{\Roman{section}}
\renewcommand{\thesubsection}{(\roman{subsection})}
\renewcommand{\thesubsubsection}{\thesubsection.arabic{subsubsection}}
\setlength\parindent{5pt}
\setlength{\columnsep}{1cm}

\begin{document}

\twocolumn[
\begin{@twocolumnfalse}
\title{\Large{\textbf{Optical Pumping of $Rb^{85}$ and $Rb^{87}$}}}
\author{Herbert D. Ludowieg}
\setlength{\droptitle}{-0.65in}
\maketitle
\begin{onecolabstract}
\justify
Optical Pumping is a spectroscopical method that was developed in the 1950's 
and has been a very accurate method to determine spectroscopical properties of 
certain materials. In this experiment the following were determined: the 
individual g-factors, nuclear spins, cross sectional area and ratio of the 
periods. For $Rb^{85}$ the g-factor and nuclear spin were found to be: 
0.3260 $\pm$ 0.0005 and 2.571 respectively. For $Rb^{87}$ they were found to 
be: 0.482 $\pm$ 0.001 and 1.576 respectively. The cross-sectional area and 
ratio of the periods were found to be: 1.8$\times10^{-16}$ $\pm$ 
0.3$\times10^{-16}$ and 1.44 $\pm$ 0.05 respectively.
\\
\end{onecolabstract}
\end{@twocolumnfalse}]

\section{Introduction}
Optical Pumping is a spectroscopical method developed in 1950 by Alfred 
Kastler, whom received the Nobel Prize in physics in 1966 for his discovery. 
This method is one in which photons are utilized to create population 
differences of electronic excited and ground states. So the meaning and general 
concept is in the name itself.
\\
Under startdard conditions the population difference required to carry out 
experiments is not possible because from statistical mechanics at thermal 
equilibrium we have an equal number of electrons that rise and fall from 
excitation levels. Due to this, they tend to cancel each others effects and 
no net population differences can be detected. This is also the basis of lasers 
where, a population difference needs to be created so that photons can be 
spontaneously emmitted by the lasing medium.
\\
For this experiment the equipment that is being used is provided by TeachSpin 
and consists of an Radio Frequency (RF) discharge lamp, Interference Filter, 
Polarizers, Quarter Wave plate, absorption cell, optical detector, three sets 
of magnetic coils in a Helmholtz configuration and a RF magnetic coil. The 
sample is a Rubidium glass bulb that contains neon gas with a pressure of 
approximately 0.04 atm pressure. The presence of the neon gas is important as 
its spherical symmetry will reduce the interactions between the Rubidium atoms 
and the outside environment. They will act as a buffer gas.
\\
Optical Pumping is a process in which has had much applicability in solid state 
and liquid state physics. However, we will only be dealing with a gas since at 
the solid and liquid phases the interactions between the neighboring atoms 
increases thus broadening the energy levels \cite{ref:1}.

\section{Theory}
\subsection{Structure of alkali atoms}
In the experiment described in this paper we will be studying the absorption 
and emission from Rubidium isotopes (85 and 87) which are alkali atoms. As such the electronic structure of Rubidium is as such,
\begin{equation*}
1s^22s^22p^63s^23p^63d^104s^24p^65s
\end{equation*}
Where we can show the shorthand version as,
\begin{equation*}
[Kr]5s
\end{equation*}
Where, the subscripts show the number of electrons contained in each of the 
electronic shells. Since the only valence electron is in the 5s shell we can 
consider the atom to be like that of a Hydrogen atom and only concentrate on 
the one electron on the fifth shell. This electron much like with other 
electrons can be described by means of the total angular momentum of the 
electron \textbf{J} where it is made of components \textbf{S} and \textbf{L}. 
Which, represent the spin angular momentum and orbital angular momentum 
respectively.
\\
Since, 

\begin{thebibliography}{9}
\bibitem{ref:1}
Bloom, A L (1960). Optical Pumping. \emph{Scientific American} October, 72.
\bibitem{ref:2}
Benumof, R (1965). Optical Pumping Theory and Experiments. \emph{American 
Journal of Physics 33}, 151.
\bibitem{ref:3}

\end{thebibliography}

\end{document}
